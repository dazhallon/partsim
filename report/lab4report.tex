\documentclass[two column]{article}
\usepackage[utf8]{inputenc}

\usepackage{amsmath,amsthm,amssymb}

\usepackage{subcaption}
\usepackage{graphicx}

\usepackage{diagbox}

\graphicspath{{./../images/}}


\usepackage{listings}


\usepackage{color}

\definecolor{gray}{rgb}{0.5,0.5,0.5}
\definecolor{orange}{rgb}{0.8,0,0}

\lstdefinestyle{ccode}{
  belowcaptionskip=1\baselineskip,
  breaklines=true,
  frame=L,
  xleftmargin=\parindent,
  language=C,
  showstringspaces=false,
  basicstyle=\footnotesize\ttfamily,
  keywordstyle=\bfseries\color{green},
  commentstyle=\color{gray},
  identifierstyle=\color{blue},
  stringstyle=\color{orange},
}

\def\listingsfont{\ttfamily} 
\def\listingsfontinline{\ttfamily}

\title{Report for computer exercise 4 - Particle simulator in MPI}
\author{Anton Karlsson\\antka388\\931217-7117}
\date{}
\begin{document}

\maketitle

\section{Code overview}
\label{sec:code-overview}

\subsection{Discussion on partitioning}
\label{sec:disc-part}



The box in which the particles was contained in was partitionion column
wise among the processors, meaning that each process owned a local box
with the orginal hieght and a width of $\texttt{BOX\_VERT\_LENGTH}/p$,
where $p$ is the number of processors. The reasoning behind this
partitioning is that is is easy to implement, however, the downside is
that the speed-up  is not the best. This is because the the local boxes
that is owned by a process, has an area that decreases fast as the
number of processes increase. 

A partitioning that might be better is to partitioning in squares, 
as the area will not dercease as fast. Thus
the probability a particle to a different process will no
decrease has fast has the column-wise partitioning. However, the
implementation of the square partitioning is harder to code.

\subsection{Execution}
\label{sec:execution}

First, the process definie their deminsion of it own local partiioning
using the global values of the original box deminsions along with their
process id. Then each process intilies equaly many particles (given by
the paramter \texttt{NUM\_PART}) with random position (within the local
box of the process) and a random velocity. This way, we avoid some
initial communcation without losing correctness.

The main loop that was given can now be executed. Each process moves
its local particles and deal with eventual collisions and wall
collsions that occour. Note that any collisions that appear between
particels that are not owned by the same process are ingored. Since the
probability of  this happening is realative small and therefore will
not affect the final result noetworthily. 

When the partilces has been moved, the process check wheter any
particles has coordinates that indicates that it need to move to a
neighboring box. The particles that is to be sent to left respective
right is sent using dynamic communication. 


\section{Results}
\label{sec:results}

The elapsed times  are presented below. Here, $p$ denotes the number
of processors used and $n$ deontes the problem size. In this
case it denotes the number of particles used in the simulation. In all of the simulations, the simulated time was 1000 seconds.

\begin{table}[h]
  \centering
  \begin{tabular}[]{c|c|c|c}
    \backslashbox{$p$}{$n$} & 1000 & 2000 & 4000 \\
    \hline 
    1 & 5.1946  &   20.042 & 82.7355 \\
    2 & 1.23806 &  4.95515 & 20.7491 \\
    4 & 0.32439 & 1.30241  &  5.07428
  \end{tabular}
  \caption{Elapsed time when differrent number of processors was used.}
  \label{tab:1}
\end{table}

\subsection{Verification of the gas law}
\label{sec:ver-of-gas-law}

The average pressures obtained using different number of particles and 
box sizes are covered in Table \ref{tab:2}. As we can we, the pressure 
changes linearily proportional to the number of particles and area (volume) 
used.

\begin{table}[h]
  \centering
  \begin{tabular}[h]{c|c|c}
    \backslashbox{$V$}{$n$}  & 2000 & 4000 \\
    \hline 
    10000 $\times$ 10000 &  0.008375 & 0.016690 \\
    15000 $\times$ 10000 &  0.005340 & 0.011739   
  \end{tabular}
  \caption{A table that verifies the gas law.}
  \label{tab:2}
\end{table}

\clearpage
\onecolumn
\section*{Code}
\label{sec:code}

\lstinputlisting[caption=main.c,style=ccode]{./../main_sequential.c}

\end{document}


%%% Local Variables:
%%% mode: latex
%%% TeX-master: t
%%% End:
